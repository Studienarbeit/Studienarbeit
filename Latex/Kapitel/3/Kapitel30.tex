\chapter{Honeynets}
Honeynets bestehen aus mehreren virtuellen oder physikalischen High-Interactive Honeypots, die ein komplettes Netzwerk darstellen\cite{grimes.2003a}\cite{spitzner.2002a}. Jede Netzwerkkomponente und jeder Server kann dabei als Honeypot angesehen werden. Es entsteht somit die Möglichkeit, ein komplettes Produktivnetz nachzustellen, welches dem Hacker den Eindruck vermittelt, in ein produktiv verwendetes Firmennetz eingedrungen zu sein\cite{grimes.2003a}.

\section{Anforderungen eines Honeynets}
\noindent Die Anforderungen eines Honeynets kann grob ich drei Kategorien unterteilt werden:\\
\\
\textbf{Data Control: }Datenkontrolle bedeutet, dass der Betreiber eines Honeypots oder Honeynets die Kontrolle der ein- und ausgehenden Datenpakete behält. Gelingt es dem Hacker dem Honeynetbetreiber diese Kontrolle zu entreißen, besteht die Möglichkeit, dass der Hacker einen Angriff auf das Produktivnetz startet. 
Die Datenkontrolle muss für den Angreifer unsichtbar sein. Zu strenge Sicherheitsvorkehrungen können den Hacker jedoch verunsichern und ihn auf den Honeypot aufmerksam machen (z.B. ausgehende Verbindungen blockieren). Jedoch kann z.B. eine Limitierung der ausgehenden Verbindungen ein gutes Mittel gegen den Verlust den Datenkontrolle sein. \\
\\
\textbf{Data Capture: }Die Informationen, die ein Hacker während eines Angriffes hinterlässt, müssen möglichst reichhaltig und unauffällig dokumentiert werden. Für die Datensammlung gibt es verschiedene Möglichkeiten die Aktionen eine Hackers aufzuzeichnen.
\begin{itemize}
\item Packet Sniffing: Zum aufzeichnen des kompletten Netzverkehrs (ein- und ausgehende Pakete).
\item Keystroke Logging: Das aufzeichnen der vom Hacker ausgeführten Tastenanschläge.
\item Snapshot Software: Vergleicht Betriebssystem vor- und nach der Kompromittierung und hält die Änderungen fest.
\item Log-Dateien wie z.B. Log-Daten eines Netzwerkgerätes wie Switch und Router
\end{itemize} 
Wichtig bei diesen Programmen ist es, dass sie für den Hacker nicht ersichtlich sein dürfen. Findet der Angreifer eines dieser Programme ist dieser alarmiert und wird die Flucht ergreifen.\\
\\
\textbf{Data Collection: }Bei einem verteilten System wie z.B. bei einem Honeynet muss es eine zentrale Stelle geben in der die Informationen gesammelt und gespeichert werden. Daten werden nie direkt auf einem Honeypot protokolliert, sondern an ein zentrales System übertragen. Wichtig hierbei ist es, dass die Daten sicher und unverändert an das System übertragen werden. Der Angreifer soll nicht die Möglichkeit haben einen Angriff zu vertuschen, oder gar das System selbst anzugreifen. \\
\\
\textbf{Data Analysis: }Die gewonnenen Informationen müssen dem Zweck entsprechend ausgewertet werden. Je nach Ziel des Honeypots müssen z.B. Gegenmaßnahmen getroffen, oder aus dem gelernten Wissen Schlüsse gezogen werden, um in Zukunft eine Kompromittierung eines Produktivsystems zu verhindern.

\section{Architektur eines Honeynets}
Es gibt zwei verschieden Arten von Architekturen die sich im Laufe der Zeit durchgesetzt haben. Diese werden in Generation I und Generation II Honeynets (Abk. GenI und GenII) unterteilt.\\
\subsection{GenI Honeynet}
Bei einem GenI Honeynet wird das gesamte Netz durch eine Firewall in drei Teile unterteilt. Der erste Teil ist das Produktivnetz indem sich das zentrale Management System befindet. Der zweite Teil ist das Internet, welcher das Zugangsmedium des Angreifers darstellt. Der dritte und letzte Teil ist das Honeynet\cite{WebGenI.2006b}. 

GenI Honeynets gelten als die ersten richtigen High-Interactive Honeypots, da sie weit mehr Informationen als normale Honeypot, und unbekannte Angriffe aufzeichnen können\cite{spitzner.2002a}.

Der Prozess der Datensammlung beginnt bereits mit passieren der Firewall. Dort können Informationen wie die verwendeten Protokolle, Zeitstempel,IP-Adressen und Ports gesammelt werden. Außerdem wird hier kontrolliert wie oft der Angreifer eine Verbindung eingehen kann (Data Control). Wie viele Versuche zugelassen werden hängt vom Verwendungszweck des Honeynets ab. Der Router zwischen Honeynet und Firewall unterstützt diese auf zwei verschiedene weisen. Zum Einem versteckt er die Firewall vor dem Hacker. Der Angreifer denkt, er greift auf einen produktiven Router zu. Zum Anderen unterstützt er die Firewall in Sachen Zugriffskontrolle. So kann ein Single-Point-of-Failure vermieden werden\cite{grimes.2003a}\cite{WebGenI.2006b}. 

Ein \acf{IDS}-System steht nun noch zwischen dem Angreifer und den Honeypots. Dieses ist meist über einen Switch (oder wie in Abb. \ref{hnet:geni} mit einem Router) mit dem gesamten Honeynet verbunden. Dort werden alle Netzwerkaktivitäten protokolliert und bei bestimmten Angriffsmustern gegebenenfalls ein Alarm ausgelöst\cite{spitzner.2002a}. 

\begin{figure}[ht]
    \centering\includegraphics[scale=0.5]{Bilder/GenI.pdf}
  \caption{GenI Honeynet}
  \label{hnet:geni}
\end{figure}

\noindent Die GenI Technologie bietet sich besonders an, um automatisierte oder Anfänger-Hacks zu erkennen. Meist handelt es sich dabei um Ziele, deren Schwachstelle zufällig entdeckt wird, und es dadurch zu einem Angriff kommt. 
Die Architektur ist nicht effektiv für fortgeschrittene Angreifer, oder Hacker, die ein bestimmtes System angreifen wollen. Zum einen sind diese relativ einfach über einen Fingerprint ausfindig zu machen, zum Anderen bestehen sie meist aus einer Standardinstallation eines Betriebssystems, weswegen sie für Angreifer meist uninteressant wirken\cite{spitzner.2002a}.\\

\noindent\textbf{Methoden zur Datenkontrolle}\\
\noindent Die Datenkontrolle eines GenI Honeypots besteht im Grunde aus der Layer 3 Firewall, die das Produktivnetz vom Honeynet trennt. Die Firewall erlaubt jeglichen Zugriff in das Netz, limitiert jedoch die ausgehenden Verbindungen (nicht Pakete). Der Firewall wird vom Administrator eine Grenze für ausgehende Verbindungen mitgeteilt, nach erreichen dieser wird jeder ausgehender Verbindungsversuch geblockt. Je mehr Verbindungen erlaubt sind, desto mehr Freiheiten hat der Hacker seine Aktionen durchzuführen. Um den Schaden zu minimieren die der Hacker verursacht, kann auch eine geringe Grenze gewählt werden. Automatisierte Attacken können so z.B. weiterhin festgestellt werden. Das Honeynet Project empfiehlt hierfür die Linux Firewall IPTables oder als kommerzielle Version FireWall-1 \cite{spitzner.2002a}.\\

\noindent\textbf{Methoden zur Datenaufzeichnung}\\
\noindent Die Datenaufzeichnung eines GenI Honeynets muss, wie zuvor definiert (vgl. Kapitel 3.1), für den Angreifer unsichtbar sein. Die Daten, die gesammelt werden, dürfen nicht lokal auf dem Honeypot gespeichert werden. Um dies zu verwirklichen werden die Daten in verschiedenen Schichten.
Die erste Schicht ist die Logging-Aktivität auf der Firewall. Alle Daten, die in das Honeynet gelange, müssen zunächst durch die Firewall. Dort können zwar keine Informationen wie Tastendrücke oder Packet-Payloads aufgezeichnet werden, jedoch können Protokoll Header, in denen sich Informationen wie Zeitpunkt, Quell- und Zieladresse sowie Quell- und Zielport befinden, ausgelesen und aufgezeichnet werden\cite{spitzner.2002a}. 

Die zweite Schicht ist das IDS-System, welches mit dem Produktivnetz und dem Honeynet jeweils mit einem Interface verbunden ist. Das Interface, dass mit dem Honeynet verbunden ist besitzt keine IP-Adresse. Es gilt als passives Interface (rote Linie in Abb. \ref{hnet:geni}), welches den kompletten Datenverkehr des Honeynets aufzeichnet, jedoch keine Angriffsfläche für den Hacker bietet, da es keine IP-Adresse besitzt. Das zweite Interface erlaubt es dem Administrator im Produktivnetz auf die gesammelten Daten zuzugreifen. Das IDS zeichnet die kompletten Datenpakete mit Payload auf und stellt diese später für eine Datenanalyse zur Verfügung. Die zweite Aufgabe des IDS ist es, eine Warnung bei ungewöhnlichen Aktivitäten zu geben.

Die dritte Schicht stellen die Honeypots selbst dar. Alle System und 
\subsection{GenII und GenIII Honeynets}
GenII Honeynets wurden entwickelt, um die Probleme der ersten Version zu beheben. Dabei soll das System allgemein leichter aufzustellen, und schwieriger zu entdecken sein. Die meisten Änderungen wurden bei der Datenkontrolle unternommen. Die Architektur eines Honeynets der zweiten Generation unterscheidet sich signifikant zu der eines GenI Honeynets. In Abb. \ref{hnet:genii} befindet sich eine Beispielarchitektur eines GenII Honeynets\cite{spitzner.2002a}\cite{WebGenII.2006b}.\\

\begin{figure}[h]
    \centering\includegraphics[scale=0.5]{Bilder/GenII.pdf}
  \caption{GenII Honeynet}
  \label{hnet:genii}
\end{figure}

\newpage
\noindent\textbf{Methoden zur Datenkontrolle}\\
\noindent GenI Honeynets erhalten ihre Datenkontrolle durch eine Limitierung der ausgehenden Verbindungen über eine Layer 3 Firewall. Das Problem hierbei war die relativ einfache Möglichkeit für den Angreifer herauszufinden, das es sich um ein Honeynet handelt (durch testen der Anzahl von ausstehenden Verbindungen oder TTL-Zähler). Die Layer 3 Firewall und das IDS werden in einem GenII Honeynet in einem Gerät, der Honeywall oder dem Honeynet Sensor, kombiniert\cite{WebGenII.2006b}. Dieser Sensor ist ein Layer 2 Gerät, ähnlich einer Bridge, welches es erschwert das Gerät ausfindig zu machen (TTL-Zähler wird nicht mehr dekrementiert, Pakete werden nicht geroutet). Jedoch wird jedes Paket welches das Honeynet empfängt oder verlässt den Sensor passieren\cite{spitzner.2002a}.
Durch die Verwendung eines Layer 2 Gerätes, befindet sich das Honeynet nicht mehr in einem separaten Netzwerk. In Abb. \ref{hnet:genii} trennt der Honeynet Sensor das Honeynet vom Produktivnetzwerk, in Wirklichkeit handelt es sich um das selbe Netz. Die Separierung der Netze findet nun auf Layer 2 statt auf Layer 3 statt.
Eine Weitere Änderung besteht in der Benutzung des IDS als Gateway. Dabei übernimmt das IDS die Funktion einer "intelligenten" Firewall, die erkennt, ob eine Verbindung legitim ist, oder ob es sich dabei um einen Angriff handelt. Zudem kann sie wie eine Firewall Verbindungen blockieren oder limitieren. Das IDS Gateway besitzt eine Signaturdatenbank, in der bekannte Angriffsmuster enthalten sind. Wir ein solches Muster erkannt, kann das IDS die Verbindung blockieren\cite{spitzner.2002a}. 
Ein Vorteil dieser Methode ist, das der Angriff nicht mehr über die ausgehenden Verbindungen identifiziert werden muss, sondern schon im voraus klar ist, was der Hacker plant. So kann verhindert werden, das der Angreifer mit seinen limitierten ausgehenden Verbindungen Schaden anrichtet. Angriffsmuster, die nicht in der Datenbank des IDS verzeichnet sind, werden als legitime Aktion angesehen. Deswegen werden meist die ausgehenden Verbindungen über das IDS weiterhin, wie in GenI, limitiert. Jedoch können hier weitaus größere Werte in Betracht gezogen werden, so dass das Fingerprinting ebenfalls erschwert wird\cite{spitzner.2002a}.
Ein anderen Vorteil dieser Vorgehensweise besteht darin, dass das GenII Honeynet auf unautorisierte Aktionen antworten kann. Versucht der HAcker von einem kompromittierten Honeypot einen Produktiv-Rechner anzugreifen, so wird das Paket vom IDS abgefangen, modifizieren um den Angriff zu neutralisieren, und eine unverständliche Antwort an den Hacker zurückschicken. Der Angreifer sieht zwar, dass sein Angriff gestartet wurde, sein Exploit ist aber nie erfolgreich.
Ein Beispiel für ein solches IDS-Gateway ist Hogwash\cite{spitzner.2002a}.\\

\noindent\textbf{Methoden zur Datenaufzeichnung}\\
\noindent Die Möglichkeiten zu Datenaufzeichnung der zweiten Generation unterscheiden sich kaum zu denen der Ersten. Die meisten Informationen werden durch das IDS gesammelt. Dateien die lokal auf den Honeypots aufgezeichnet werden, werden weiterhin auf einem Log-Server zusätzlich gespeichert. Die Probleme der ersten Generation bleiben hier vorerst weiterhin bestehen. Bei verschlüsselten Kommunikationen und Ausschalten der syslog-Funktion, bleiben nur die Daten, die über das IDS gesammelt werden können. Diese Probleme werden in der dritten Generation behoben\cite{spitzner.2002a}.\\
\newpage

\noindent\textbf{Methoden zur Datensammlung}\\
\noindent Die Datensammlung spielt eine Rolle, wenn mehrere Honeynets Daten zu einer zentralen Datensammelstelle senden. Für einfache Honeynets findet die Datensammlung auf einem Log-Server innerhalb des Honeynets statt. Für verteilte Honeynets muss ein zentraler Log Server für alle Honeynets bereitstehen. Wichtig für diesen ist, dass die Daten sicher und unverändert vom Honeynet auf das System gelangen (z.B. über einen IPSec Tunnel vom Honeynet zum zentralen Log-Server). 
Ein anderer Aspekt der beachtet werden muss ist der, dass die Daten die gesammelt werden standardisiert werden müssen. Daten die von verschiedenen Honeynets stammen sollten alle das gleiche Format haben und eindeutig einem Honeynet zuweisbar sein\cite{spitzner.2002a}. \\

\noindent\textbf{GenIII Honeynets und Zukunft}\\
\noindent Ende 2004 wurde die vorerst letzte Generation von Honeynets vorgestellt. Ein GenIII Honeynet besitzt die selbe Netzwerkarchitektur wie dessen Vorgänger, behebt jedoch einig dessen Schwachstellen. Bei dem Versuch einen Honeynet Standard zu schaffen, und eine Möglichkeit zu finden, ein Honeynet leichter zu erstellen, hat das Honeynet-Project eine CD entwickelt, die alle Anforderungen eines Honeynets beinhaltet. Diese CD (\emph{Roo} genannt) wird als dritte Generation angesehen. Die aktuelle Version (1.4, Stand: 2014) bietet neben der verbesserten Datenaufzeichnung, eine grafische Web-Oberfläche zur Datenanalyse, und unterstützt weiter Tools wie Sebek und Hflow2 (Siehe Kapitel Tools)\cite{WebGenIII.2006b}.

