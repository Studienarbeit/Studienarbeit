\subsection{GenII und GenIII Honeynets}
Die zweite Generation von Honeynets verwendet ein Gateway, welches die Funktionen der in GenI verwendetet Komponenten enthält, und diese noch weiter ergänzt. Dieses Gateway (meist Honeywall genannt) besteht nicht wie in GenI aus einem Layer-3 Router sondern aus einer Layer-2 Bridge. Dies verhindert z.B. dass der Angreifer über die TTL-Zähler eines IP-Paketes dn Router erkennen würde. Alle zuvor genannten Anforderungen werden in der Honeywall erfüllt. Für die Datenkontrolle werden hier wie in GenI die Verbindungen limitiert (oft mit dem Programm IPTables), um so DOS-Angriffe zu vermeiden und die Kontrolle über die Verbindungen zu erhalten. Zusätzlich bietet sich nun auch die Möglichkeit Zugriffe die über bestimmte Protokolle einzeln zu limitieren.

Ein IDS oder IPS verhindert weiterhin das der Hacker vom Honeynet aus einen Angriff auf das Produktivsystem oder in das Internet starten kann. Die Zugriffskontrolle und das IDS sammeln wie beim GenI Honeypot die Daten. In Abb. \ref{hnet:genii} befindet sich ein Beispiel einer GenII Honeynet Architektur.
\\
\begin{figure}[]
    \centering\includegraphics[scale=0.5]{Bilder/GenII.pdf}
  \caption{GenII Honeynet}
  \label{hnet:genii}
\end{figure}
\\
\noindent Ende 2004 wurde die vorerst letzte Generation von Honeynets vorgestellt. Ein GenIII Honeynet besitzt die selbe Netzwerkarchitektur wie dessen Vorgänger, behebt jedoch einig dessen Schwachstellen. Bei dem Versuch einen Honeynet Standard und eine Möglichkeit zu finden, ein Honeynet leichter zu Erstellen, veröffentlichte das Honeynet-Project (www.honeynet-project.org) eine CD, die alle Anforderungen eines Honeynets beinhaltet. Diese CD (\emph{Roo} genannt) wird als dritte Generation angesehen. Die aktuelle Version (1.4, Stand: 2014) bietet neben der verbesserten Datensammlung ,eine grafische Web-Oberfläche zur Datenanalyse, und unterstützt weiter Tools wie Sebek und Hflow2 (Siehe Kapitel Tools).