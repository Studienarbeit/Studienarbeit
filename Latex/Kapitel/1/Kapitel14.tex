\section{Umgebung der Arbeit}
Die Studienarbeit \glqq Analyse und Implementierung eines Honeypots-Systems\grqq wird im Rahmen des Bachelor Studiengangs Informationstechnik an der DHBW-Karlsruhe durchgeführt. Ausgearbeitet wird das Thema von den Studenten Julian Kühn und Steffen Kurstak. Für die Durchführung dieser Arbeit steht ein Server mit zwei Netzwerk-Ports zur Verfügung sowie zwei öffentliche IP-Adressen der DHBW-Karlsruhe. Das Betriebssystem des Servers wird voraussichtlich auf Linux basieren.

Neben dem Server wird auch der kauf eines vorinstallierten Honeypots angestrebt, welcher damit ebenfalls zur Verfügung stehen wird

Die Dokumentation dieser Arbeit wird in \LaTeX geschrieben. Für einen vereinfachten Umgang mit \LaTeX wird die Software TeXstudio eingesetzt. Ein ausschlaggebendes Kriterium für den Einsatz von \LaTeX, ist der unkomplizierte Umgang bei einem parallelen Arbeiten der Autoren. Durch die Untergliederung in einzelne Dateien werden Konflikte vermieden. Mit Hilfe des Repositorys GitHub wird die Versionsverwaltung vorgenommen. 