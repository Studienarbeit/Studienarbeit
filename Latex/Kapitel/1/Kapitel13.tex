\section{Stand der Technik}
Zwischen 2000 und 2006 war der Honeypot am populärsten. Neben Firmen haben auch Privatnutzer mit Hilfe der Honeypots angefangen Hacker zu studieren. Da ein Honeypot allerdings gerade in der Industrie nur Geld kostet und oftmals keinen messbaren Mehrwert bringt, ist der Einsatz von Honeypots zurückgegangen. Zudem sind auch die Communitys zurückgegangen und der damit einhergehend Support von OpenSource Projekten. Die Honeyport-Software Honeyd ist hierfür ein passendes Beispiel, da das letzte Release der Software bereits aus dem Jahre 2007 ist. Somit ist bei dem Einsatz von OpenSource Produkten in dieser Studienarbeit, mit Problemen bezüglich Dokumentation und Support zu rechnen. Hilfreich wäre hier eine kommerzielle Lösung, um diesen Problemen aus dem Weg zu gehen. 