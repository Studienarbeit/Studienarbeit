\subsection{Abschließende Schlussfolgerung}
Nachdem die Datenanalyse erfolgreich war, ist festzulegen welche Informationen interessant waren, oder welche bereits bekannt. Meist kann nach folgenden Kriterien ein Angriff klassifiziert werden:

\begin{itemize}
\item Wurde eine andere Angriffsmethode verwendet (im Vergleich zu vorhergehenden Angriffen und Analysen)
\item Welche Verschlüsselung wurde verwendet
\item Wurde ein unübliches Protokoll verwenden
\item War der Angriff auf den Honeypot gerichtet oder war es Zufall
\item Was kann aus dem Angriff gelernt werden
\item Wie kann in Zukunft ein solcher Angriff verhindert werden (falls es sich um einen neuen handelt)
\end{itemize}

\noindent Nach einer Analyse werden üblicherweise Änderungen am Honeypot vorgenommen, um einen wiederholten Angriff dieser Art zu vermeiden. Nachdem alle wichtigen Schlüsse gezogen wurden, muss entschiedene werden wie der Honeypot als nächstes verwendet werden soll. Der Honeypot sollte komplett neu aufgesetzt werden, und kann nun je nach nächsten Verwendungszweck modifiziert werden. Nach einem erneuten Angriff beginnt der Analyseprozess erneut.
