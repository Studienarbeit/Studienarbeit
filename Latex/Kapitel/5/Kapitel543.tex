\subsection{Analyse des Dateisystems}
Bei der Analyse des Dateisystems kann festgestellt werden welche Dateien hinzugefügt, gelöscht oder verändert wurden. Außerdem kann nach Daten gesucht werden die nicht in Dateien gespeichert wurden (z.B. in unbenutzten Regionen der Festplatte). Eine einfache Methode dies zu bewerkstelligen wäre das Vergleichen des \emph{"zuletzt geändert"} Attribut jedes Ordners bzw. jeder Datei. Programme wie AFind listen alle Daten bezüglich des Veränderungsdatums auf. Dabei ist zu beachten das das System selbst manche Daten ändert und sich das Datum so ebenfalls ändern kann. Diese Methode ist jedoch nur begrenzt effektive, da es allgemein relativ leicht für Hacker möglich ist, die Systemzeit oder ein Zeitstempel einer Datei oder eines Verzeichnisses zu manipulieren. 

Eine effektivere Lösung bietet das \emph{"hashen"} von Dateien. Beim hashen werden große Ursprungsdaten auf eine kleine Datenmenge abgebildet. Ändert sich die eigentliche Datei nur minimal, ändert sich auch der darüber gebildete Hashwert. So kann, wenn man über den ursprüngliche Hashwert verfügt, über einen Vergleich des alten und des neuen Hashwerts überprüfen, ob sich eine Datei verändert hat. 
Über jede Datei des Honeypots sollte vor Produktivschaltung ein Hashwert gebildet werden. Dies kann über verschiedenen kostenfreie Anwendungen stattfinden. Ein Beispiel dafür ist FileCheckMD5. Dieses Scannt rekursiv Verzeichnisse und bildet Hashwerte über die beinhalteten Dateien. Die Werte werden dabei gespeichert und können für die spätere Überprüfung verwendet werden.
