\section{Definition von Hacking}
Unter Hacking wird oft das gesetzwidrige aushebeln von Programmfunktionen um einer illegalen Aktivität nachzugehen, verstanden. Laut (Jon Erickson) handelt es sich jedoch um "das Auffinden von unbeabsichtigten oder übersehenen Anwendungsmöglichkeiten von Regeln, die auf eine neue und originelle Weise angewendet werden, um ein Problem zu lösen, wie immer dieses auch sein mag." 
Ob ein Hack nun bösartig ist oder nicht, hängt vom ausführenden Hacker und dessen Absichten ab. Deswegen unterscheidet man im allgemeinen unter verschiedenen Typen von Hackern:\\

\noindent\textbf{White-Hat}\\
Als White-Hat Hacker werden jene bezeichnet, die sich mit ihren Aktionen im gesetzlichen Rahmen befinden. Sie verwenden ihr wissen, um Sicherheitslücken zu erkennen, und andere daraufhin aufmerksam zu machen. Sie werden oft von größeren Firmen eingesetzt, um deren Systeme mit professionellen Penetrationstests auf Schwachstellen zu überprüfen. \\

\noindent\textbf{Black-Hat}\\
Black-Hat Hacker wollen durch ihre Hacking Angriffe schaden verursachen. Dabei kann es sich z.B. um Datendiebstahl, Systembeschädigungen oder das Blockieren von Diensten handeln. Dies ist die Art von Hackern, die man letztendlich mit einem Honeypot anlocken möchte.\\

\noindent\textbf{Grey-Hat}\\
Diese Form von Hackern ist eine Mischform der zuvor genannten. Sie halten sich womöglich nicht in eine gesetzlichen Rahmen, verfolgen jedoch mit ihren Aktionen meist ein höheres Ziel, wie z.B. Firmen zu Veröffentlichung von Informationen zu bekannten Sicherheitslücken zwingen.\\

Neben diesen Kategorien gibt es noch die s.g. Scriptkiddies, welche meist aus Jugendlichen besteht, die mit wenigen Grundkenntnissen vorgefertigte Automatismen verwenden, um in Systeme einzudringen oder Schaden anzurichten. Script-basierte Angriffe werden meistens von Honeypots registriert, da sie oft versuchen auf Dienste zuzugreifen, die das System nicht unterstützt.