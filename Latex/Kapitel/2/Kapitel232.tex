\subsection{Datenschutz}
Ist ein Angreifer auf einem Honeypot zu Gange, wird er oftmals von Tools überwacht und aufgezeichnet. Diese Aufzeichnungen stehen dem Betreiber zur Verfügung und dass ohne das Wissen des Angreifers. Dornseif zweifelt dabei die Unwissenheit des Angreifers an und argumentiert damit, dass die blackhat-Community über die Existenz von Honeypots informiert sind. Somit nehmen sie bei ihren Angriffen eine Aufzeichnung und Überwachung in Kauf. 

Hinzu kommt, dass bei den Angriffen von blackhats keine personenbezogenen Daten in die Hände des Betreibers fallen. Mit dieser Argumentation kann auch der Datenschutzrechtliche Hintergrund entkräftet werden.	