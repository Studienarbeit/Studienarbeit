\subsection{Zivil- und Strafrecht}
Sollte der Honeypot, von einem blackhat, für einen Übergriff auf einen unbeteiligten Dritten genutzt werden, könnte ihm Strafrechtlich eine Beihilfe vorgeworfen werden. Um eine Beihilfe einer Straftat zu leisten, erklärt \cite{dornseif.2012a}, muss eine aktive Hilfeleistung seitens des Honeypts-Betreibers nachgewiesen werden.

Im Fall der Honeypots ist dieser Nachweis nicht möglich, da es gezielt Maßnahmen zum Schutz vor Übergriffen durch den Honeypot auf fremde Systeme gibt. Der Sicherheitsstandard eines Honeypots ist oft nicht geringer als der anderer Systeme. Zudem kann auch nicht behauptet werden, dass sich ein Honeypot einem blackhat explizit aufdrängt oder ihn zu einer Straftat verleitet.

Zivilrechtlich stellt sich nun die Frage, ob der Betreiber für einen aufgekommen Schaden Haftbar gemacht werden kann. \cite{dornseif.2012a} beantwortet diese Frage ebenfalls mit nein. Um den Betreiber Haftbar zu machen, muss der Honeypot aktiv anderen Systemen schaden zufügen. Da ein solcher Sachverhalt nicht vorliegt, bleibt die Möglichkeit einer Unterlassung der Absicherung. Diese Anschuldigung stützt sich auf der sogenannten Verkehrssicherungspflicht. Hierbei geht es im groben darum, dass der Betreiber die Gefahren, die bewusst durch den Honeypot geschaffen werden, mit angemessenen Vorkehrungen gering hält. Somit soll der schaden Dritter vermieden werden.

Auch hier ist zu sagen, dass Honeypots ausreichend überwacht und gesichert werden, wodurch eine Unterlassung der Absicherung ausgeschlossen werden kann.