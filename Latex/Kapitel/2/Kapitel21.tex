\section{Definition}

Einleitend und noch oberflächlich erklärt ist die Aufgabe eines Honeypots, einen Angreifer von einem bedeutsamen Ziel abzulenken. In der Informationstechnik erfüllen Honeypots, je nach Einsatzgebiet und Ansicht des Entwicklers, verschiedene Aufgaben.

Einige Unternehmen sehen den Einsatz eines Honeypots als ein Intrusion Detection System. Für andere ist es nur eine Täuschung für Hacker, welche dadurch von den produktiven Systemen abgelenkt werden sollen. Ein weiterer Nutzen eines Honeypots ist es, die angreifenden Hacker analysieren zu können und somit neue Vorgehensweisen und Trends der Hacker zu erkennen. Diese Informationen sind vor allem für Sicherheitsfirmen relevant, da sie dabei neue Viren, Trojaner oder weitere Schadsoftware erkennen können.

Trotz dieser verschiedenen Einsatzmöglichkeiten hat Lance Spitzner eine passende Definition gefunden: \glqq A honeypot is security resource whose value lies in being probed, attacked, or compromised.\grqq \cite{spitzner.2002a}

Der Honeypot ist nach Spitzner eine Sicherheits-Ressource deren Wert darin liegt, erforscht, attackiert und kompromittiert zu werden. Durch diese Definition ist offen gelassen, welchen Nutzen der Anwender daraus zieht. Mit dieser allgemein gehaltenen Definitionen können nun Ziele eines Honeypots im nächsten Kapitel beschrieben werden.
