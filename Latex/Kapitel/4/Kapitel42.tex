\section{Risiken}
Da ein Honeypot oder Honeynet von echten Angreifern attackiert wird, muss damit gerechnet werden das der Angreifer eventuell versucht über den Honeypot in das Produktive Netz zu gelangen. Es muss dringend vermieden werden das eine solche Möglichkeit besteht, indem das/die Hostsystem/e genügend abgesichert wird. Außerdem muss vermieden werden, dass der Angreifer weiter Angriffe über das Internet vom Kompromittierte Netz aus startet (z.B. durch limitierte Verbindungsversuche aus dem Netz hinaus). \\
\\
Neben den Risiken die eine Honeypot für das eigene Netz bringt, können auch unbewusst einige rechtliche Gesetzte verletzt werden (laut Strafgesetzbuch):\\
\\
\emph{§ 26. Anstiftung. Als Anstifter wird gleich einem Täter bestraft, wer vorsätzlich einen anderen zu dessen vorsätzlich begangener rechtswidriger Tat bestimmt hat.}\\
\\
\emph{§27. Beihilfe. (1) Als Gehilfe wird bestraft, wer vorsätzlich einem anderen zu dessen vorsätzlich begangener rechtswidriger Tat Hilfe geleistet hat. (2) Die Strafe für den Gehilfen richtet sich nach der Strafdrohung für den Täter. Sie ist nach § 49 Abs. 1 zu mildern.}\\
\\
Über einen Emulierten Honeypot wäre eine Straftat schwer zu verwirklichen, da jedoch in Deutschland schon der versuch einer Straftat als Verbrechen gilt könnten sich dies in diesem Fall als problematisch erweisen.
Des weiteren könnte ein absichtlich schlecht gewartetes Betriebssystem (z.B. aus Research-Gründen), das von einem Angreifer kompromittiert wurde, als Plattform für weitere Angriffe genutzt werden. Da das Betriebssystem vorsätzlich ungesichert ist, würde dies als Beihilfe für eine Straftat gelten.  

 