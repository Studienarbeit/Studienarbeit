\subsection{Low-Interactive}
Ein Low-Interaction Honeypot bietet im Vergleich zu einem High-Interactive Honeypot keine "reale" Umgebung. Meist emuliert ein Programm (z.B. honeyd) einen oder mehrere Dienste die dem Hacker einen eher eingeschränkten Zugang auf ein System gewähren. So wird meist auch nur ein Zugriff auf die Netzwerkschicht des OSI-Modells und auf gewünschte Ports zugelassen. Da es sich bei dieser Implementierungsart um eine Emulation bestimmter Angriffsschnittstellen handelt muss nach einer versuchten Kompromittierung auch das Betriebssystem nicht vollständig neu  aufgesetzt werden. Dadurch wird der Wartungsaufwand im Vergleich zu High-Interactive Honeypots erheblich vermindert.

Bei einem versuchten Angriff durch einen Wurm oder einem Hacker wird z.B. nur eine Verbindungsanfrage zugelassen, jedoch nicht akzeptiert. Dadurch kann z.B. über einem Portscanner die Herkunft der Verbindungsanfrage ermittelt werden. Das forensische Material das bei Low-Interactive Honeypots gesammelt werden kann ist deswegen im Vergleich zu High-Interactive Honeypots eingeschränkt. 