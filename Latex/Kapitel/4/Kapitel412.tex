\subsection{High-Interactive}
High-Interactive Honeypots bieten die meisten Daten über Angreifer und deren Verhalten, erfordern jedoch auch die meiste Arbeit bei der Installation und Wartung. Außerdem stellen sie auch das höchste Risiko für das Produktivnetz dar. Das Ziel eines High-Interactive Honeypots besteht darin, dem Hacker eine Betriebssystemumgebung zur Verfügung zu stellen, in dem er ohne Einschränkungen seine Arbeit verrichten kann. Dabei kann das komplette Nutzerverhalten des Hackers aufgezeichnet werden, welche Schwachstellen er ausnutzt oder wie er mit anderen Kommuniziert\cite{spitzner.2002a}. 

Ein High-Interactive Honeypot ist im Prinzip das Selbe wie ein Produktivgerät. Der einzige Unterschied besteht darin, dass er keine Bedeutung für das eigentliche Betriebsnetz besitzt, sondern einzig und allein für den Zweck erstellt wird, angegriffen zu werden. Diese Art der Implementierung besitzt deswegen den höchsten Risikograd. Dem Angreifer steht das komplette Betriebssystem zur Verfügung, und kann darüber versuchen auf das eigentliche Produktivnetz überzugreifen oder Produktionsaktivitäten aufzunehmen. Dies zu verhindern benötigt einen stark erhöhten Aufwand im Verglich zu Low- oder Medium-Honeypots\cite{spitzner.2002a}. 

Die Zugriffskontrolle findet meist über eine Firewall statt. Meist erlaubt die Firewall zwar den Zugriff von außen auf den Honeypot, verbietet danach jedoch jeden Kommunikationsversuch zurück nach draußen. Dies erschwert es dem Angreifer wirklichen Schaden anzurichten, macht ihn jedoch meist zeitgleich darauf aufmerksam, dass etwas nicht stimmt. Ein Großteil der Arbeit geht dabei für die Erstellung eines passenden Regelwerks für die Firewall ein\cite{spitzner.2002a}.

Um alle relevanten Daten aufzeichnen zu können wird meist noch ein IDS benötigt. Der Wartungsaufwand eine High-Interactive Honeypot erhöht sich stark durch die ständige Neukonfiguration von Firewall Regelwerk und IDS Datenbank. Zudem muss der Honeypot nach jeder erfolgreichen Kompromittierung neu aufgesetzt werden, um dem nächsten Angreifer eine unverränderte neue Arbeitsfläche zu garantieren\cite{spitzner.2002a}\cite{grimes.2003a}.

Dieser Aufwand wird jedoch mit den ausführlichsten Daten über den Hacker und dessen Vorgehensweise belohnt. Ein "richtigen" High-Interactive Honeypot ist nur durch ein Honeynet realisierbar\cite{spitzner.2002a}.