\subsection{Virtualisierung}
Honeypots können wie normale Betriebssysteme über eine Virtualisierungssoftware auch virtualisiert werden. Dabei werden alle Dienste, Ports und OSI-Schichten emuliert. Zur Realisierung dieser Implementierungsart gibt es verschieden Möglichkeiten.\\

\noindent\textbf{Virtual Machine Honeypots}\\
Hierbei werden reale Betriebssysteme über eine Virtualisierungssoftware (z.B. VMWare oder VirtualBox) auf einem Rechner installiert. So ist es Möglich, auf einem physikalischen Rechner oder Server mehrere Honeypots in Betrieb zu halten. Der Vorteil dieser Methode ist, dass dem Hacker ein vollständiges OS mit allen Diensten, Netzwerkschichten und Anwendungen zur Verfügung gestellt wird, und trotzdem sich der Implementierungsaufwand in grenzen hält. Wird ein System kompromittiert, so kann es ohne viel Aufwand wiederhergestellt werden. Durch die Möglichkeit mehrerer Virtuelle Honeypots auf einem physikalischen Server zu betreiben ist es Möglich, ein komplettes virtuelles Honeynet auf einem Server zu erstellen\cite{grimes.2003a}.  \\

\noindent\textbf{Emulated Honeypots}\\
Um einen Honeypot zu emulieren wird meist eine Software verwendet, die die grundlegenden Funktionen eines Betriebssystems oder Dienstes emuliert. Je nach Konfiguration kann ein emulierter Honeypot auch alle Dienste, OSI-Netzwerkschichten und Applikationen zur Verfügung stellen. Die Wiederherstellung des Systems ist dabei ähnlich einfach wie die der Virtuellen Honeypots. Durch die eingeschränkten Möglichkeiten der Emulation von verschiedenen Diensten oder Systemen werden emulierte Honeypots als Low-Interactive eingestuft. Emulated Honeypots eigenen sich besonders für Honeypot Einsteiger, da sie meist einfach zu Konfigurieren sind und so schnell erste, zumeist automatisierter, Kompromittierungsversuche entdecken. Viele Tools wie z.B. honeyd arbeiten mit diesem Prinzip\cite{spitzner.2002a}.
\newpage