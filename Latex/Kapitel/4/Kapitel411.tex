\subsection{High-Interactive}
Bei High-Interactive Honeypots werden alle Dienste und Software "real" verwendet. Alle Anwendungen, Ports und alle Schichten des OSI-Modells werden dem Angreifer zur Verfügung gestellt. Diese Variante wird häufig dazu verwendet, um manuellen Hackern das Gefühl zu geben, in einen echten Server oder Computer eingedrungen zu sein. Die Überwachungsroutinen befinden sich dabei meist im Betriebssystem Kernel, um ein Entdecken des Honeypots zu erschweren. Der Vorteil dieser Methode ist, dass der Hacker alle Möglichkeiten besitzt, sein Ziel zu erreichen. Somit kann das meiste forensische Material gesammelt werden. Jedoch hat diese vollkommene Freiheit des Hackers den Nachteil, dass es für ihn leichter ist, in das Produktivsystem einzudringen. 

Diese Art der Implementierung findet sich meist bei der Anwendung von Honeynets (vlg. Kapitel 2). Der Implementierungsaufwand bei High-Interaction Honeypots ist wesentlich höher als bei anderen Implementierungsarten, da nach einer erfolgreichen Kompromittierung das komplette System neu aufgesetzt werden muss.

