\section{Typen von Honeypots}
Die Funktionsweise von Honeypots ist von der Implementierungsart abhängig. Dabei spielt der gewünschte Grad der Interaktion, sowie der Zweck des Honeypots eine Rolle (Research oder Production). Je nach Anwendungsgebiet wird die dafür nützlichste Implementierung verwendet. Grundlegend wird dabei zwischen folgenden Implementierungsmöglichkeiten unterschieden:\\

\renewcommand{\arraystretch}{1.5}
\begin{table}[h]
\caption{Interaktionsgrade mit den zugehörigen Eigenschaften\cite{spitzner.2002a}}
\label{interTab}
\begin{center}
\begin{tabular}{|c|c|c|c|c|}\hline
   \textbf{Interaktions-} & \textbf{Installation und} & \textbf{Aufstellung und} & \textbf{Informations-} & \textbf{Risiko}\\ 
   \textbf{ möglichkeit}  & \textbf{Konfiguration}    & \textbf{Wartung}         & \textbf{sammlung}      &            \\ \hline  
   \textbf{         Low}  & Einfach                   & Einfach                  & Limitiert              & Gering\\ \hline
   \textbf{      Medium}  & Kompliziert               & Kompliziert              & Variabel               & Medium\\ \hline
   \textbf{        High}  & Schwierig                 & Schwierig                & Umfangreich            & Hoch\\ \hline
\end{tabular}
\end{center}
\end{table}

\noindent Wie in Tabelle \ref{interTab} zu erkennen ist, steigt das Risiko mit erhöhter Interaktionsstufe, jedoch steigt auch gleichzeitig die Menge an Daten die gesammelt werden können. Im Folgendem wird genauer auf die verschiedenen Implementierungsarten eingegangen.
\newpage

\subsection{Low-Interactive}
Low-Interaction Honeypots sind meist am einfachsten zu Installieren, Konfigurieren und zu Warten. Sie emulieren nur bestimmte Services. Dem Hacker wird beim Zugriff eines Services der korrekte Banner angezeigt, kann seine Login-Versuche durchführen (ev. über Brute Force Angriffe), wird jedoch nie erfolgreich sein, da es sich um kein reales Betriebssystem handelt. Die Vorgehensweise des Hacker bei der Anmeldung kann so aufgezeichnet werden. 

Das primäre Ziel eines Low-Interaction Honeypots besteht darin, unautorisierte Port-Scans oder Verbindungsversuche zu entdecken. Ein Low-Interactive Honeypot wird meist über eine Anwendung emuliert (z.B. honeyd), und kann somit leicht auf einem Host-System installiert werden. Da es sich bei dieser Implementierungsart um eine Emulation bestimmter Angriffsschnittstellen handelt muss nach einer versuchten Kompromittierung auch kein Betriebssystem vollständig neu aufgesetzt werden. Dadurch wird der Wartungsaufwand im Vergleich zu High-Interactive Honeypots erheblich vermindert.

Low-Interaction Honeypots haben durch die zuvor genannten Gegebenheiten auch das geringste Risiko. Da kein wirkliches Betriebssystem dem Hacker zur Verfügung gestellt wird, kann dieser nicht auf andere Rechner zugreifen. 

Die Informationssammlung eines Low-Interactive Honeypots ist im Vergleich zu den anderen Interaktionsgraden gering. Die Informationen beschränken sich auf die Services, auf die der Angreifer versucht zuzugreifen. Der genaue Angriff sowie das eigentliche vorgehen des Hackers werden nicht erkannt. Die wichtigsten Informationen, die ein Low Interactive Honeypot sammeln kann sind\cite{spitzner.2002a}:

\begin{itemize}
\item Datum und Zeitpunkt des Angriffes
\item Quellport und Quelladresse des Angreifers
\item Ziel-IP Adresse und Zielport des Angreifers
\end{itemize}

\noindent Welche weitere Aktionen aufgezeichnet werden können hängt vom verwendeten Low-Interactive Honeypot ab.
Low-Interactive Honeypots wurden für bekannte Angriffsmuster entwickelt. Für unbekannte Vorgehensweisen eignet sich dieser nicht. 



\subsection{Medium-Interactive}
Medium-Interactive Honeypots bieten dem Angreifer mehr Freiheiten als Low-Interactive Honeypots, jedoch weniger als High-Interactive Honeypots. Ein Medium-Interactive Honeypot emuliert ähnlich wie ein Low-Interactive Honeypot Dienste, mit denen sich der Angreifer verbinden kann. Diese geben nun jedoch die korrekte Funktionalität des emulierten Systems wieder\cite{spitzner.2002a}. 

Zum Beispiel ein Microsoft IIS Web Server bietet alle Möglichkeiten der Interaktion, die ein wirklicer IIS Server zur Verfügung stellen würde. Ein Low-Interacitve Honeypot würde in diesem Fall nur den HTTP-Banner nach Zugriff auf den HTTP-Port anzeigen. Der Angreifer hat nun die Möglichkeit z.B. Trojaner, Würmer oder Viren hochzuladen. Diese können für spätere Analysen verwendet werden. Eine Gefahr besteht für das System nicht, da dem Hacker wieder kein vollständiges Betriebssystem zur Verfügung gestellt wird\cite{spitzner.2002a}.

Eine andere Möglichkeit zur Erstellung eines Medium-Interactive Honeypot besteht darin, die Funktion jail oder chroot von Unix zu benutzen. Dabei wird ein Betriebssystem partitioniert, indem eine virtuelle Betriebssystemumgebung geschaffen wird, die von einem echten Betriebssystem kontrolliert wird. Ziel ist es nun den Angreifer auf das virtuelle Betriebssystem zu locken, und ihm vom echten Betriebssystem aus zu beobachten\cite{spitzner.2002a}. \\

Medium-Interactive Honeypots haben jedoch einige Probleme. Die Komplexität der Honeypots steigt mit den Möglichkeiten diese zu Konfigurieren. Ein komplettes System zu emulieren  und richtig zu konfigurieren erfordert sehr gute Kenntnisse des Systems selbst. Je besser dieses emuliert wird, desto leichter kann es dem Hacker fallen auf das Host-System überzugreifen, was das Risiko eines Medium-Interactive Honeypot erhöht\cite{spitzner.2002a}.
Medium-Interactive Honeypots benötigen meist mehr Zeit für die Installation und Konfiguration, da mehrere Möglichkeiten beachtet werden müssen. Zudem gibt es keine fertigen Honeypot-Systeme zur Installation. Einen kompletten Microsoft IIS Web-Server zu emulieren erfordert einen sehr hohen Aufwand\cite{spitzner.2002a}.
Jedoch erhöht sich im Vergleich zu Low-Interactive Honeypots die Anzahl der gesammelten Daten erheblich. Statt nur Verbindungsversuche kann nun die tatsächliche Payload von Paketen (wie z.B. Würmer) und Benutzeraktivitäten aufgezeichnet werden\cite{spitzner.2002a}.

Der Medium-Honeypot besitzt zwar ein größeres Risiko und einen größeren Erstellungsaufwand, jedoch besitzt er auch die Möglichkeit wichtige Informationen bezüglich des eigentlichen Angriffsversuchs des Hackers zu erkennen\cite{spitzner.2002a}.



\subsection{High-Interactive}
High-Interactive Honeypots bieten die meisten Daten über Angreifer und deren Verhalten, erfordern jedoch auch die meiste Arbeit bei der Installation und Wartung. Außerdem stellen sie auch das höchste Risiko für das Produktivnetz dar. Das Ziel eines High-Interactive Honeypots besteht darin, dem Hacker eine Betriebssystemumgebung zur Verfügung zu stellen, in dem er ohne Einschränkungen seine Arbeit verrichten kann. Dabei kann das komplette Nutzerverhalten des Hackers aufgezeichnet werden, welche Schwachstellen er ausnutzt oder wie er mit anderen Kommuniziert\cite{spitzner.2002a}. 

Ein High-Interactive Honeypot ist im Prinzip das Selbe wie ein Produktivgerät. Der einzige Unterschied besteht darin, dass er keine Bedeutung für das eigentliche Betriebsnetz besitzt, sondern einzig und allein für den Zweck erstellt wird, angegriffen zu werden. Diese Art der Implementierung besitzt deswegen den höchsten Risikograd. Dem Angreifer steht das komplette Betriebssystem zur Verfügung, und kann darüber versuchen auf das eigentliche Produktivnetz überzugreifen oder Produktionsaktivitäten aufzunehmen. Dies zu verhindern benötigt einen stark erhöhten Aufwand im Verglich zu Low- oder Medium-Honeypots\cite{spitzner.2002a}. 

Die Zugriffskontrolle findet meist über eine Firewall statt. Meist erlaubt die Firewall zwar den Zugriff von außen auf den Honeypot, verbietet danach jedoch jeden Kommunikationsversuch zurück nach draußen. Dies erschwert es dem Angreifer wirklichen Schaden anzurichten, macht ihn jedoch meist zeitgleich darauf aufmerksam, dass etwas nicht stimmt. Ein Großteil der Arbeit geht dabei für die Erstellung eines passenden Regelwerks für die Firewall ein\cite{spitzner.2002a}.

Um alle relevanten Daten aufzeichnen zu können wird meist noch ein IDS benötigt. Der Wartungsaufwand eine High-Interactive Honeypot erhöht sich stark durch die ständige Neukonfiguration von Firewall Regelwerk und IDS Datenbank. Zudem muss der Honeypot nach jeder erfolgreichen Kompromittierung neu aufgesetzt werden, um dem nächsten Angreifer eine unverränderte neue Arbeitsfläche zu garantieren\cite{spitzner.2002a}\cite{grimes.2003a}.

Dieser Aufwand wird jedoch mit den ausführlichsten Daten über den Hacker und dessen Vorgehensweise belohnt. Ein "richtigen" High-Interactive Honeypot ist nur durch ein Honeynet realisierbar\cite{spitzner.2002a}.
\subsection{Virtualisierung}
Honeypots können wie normale Betriebssysteme über eine Virtualisierungssoftware auch virtualisiert werden. Dabei werden alle Dienste, Ports und OSI-Schichten emuliert. Zur Realisierung dieser Implementierungsart gibt es verschieden Möglichkeiten.\\

\noindent\textbf{Virtual Machine Honeypots}\\
Hierbei werden reale Betriebssysteme über eine Virtualisierungssoftware (z.B. VMWare oder VirtualBox) auf einem Rechner installiert. So ist es Möglich, auf einem physikalischen Rechner oder Server mehrere Honeypots in Betrieb zu halten. Der Vorteil dieser Methode ist, dass dem Hacker ein vollständiges OS mit allen Diensten, Netzwerkschichten und Anwendungen zur Verfügung gestellt wird, und trotzdem sich der Implementierungsaufwand in grenzen hält. Wird ein System kompromittiert, so kann es ohne viel Aufwand wiederhergestellt werden. Durch die Möglichkeit mehrerer Virtuelle Honeypots auf einem physikalischen Server zu betreiben ist es Möglich, ein komplettes virtuelles Honeynet auf einem Server zu erstellen.  \\

\noindent\textbf{Emulated Honeypots}\\
Um einen Honeypot zu emulieren wird meist eine Software verwendet, die die grundlegenden Funktionen eines Betriebssystems oder Dienstes emuliert. Je nach Konfiguration kann ein emulierter Honeypot auch alle Dienste, OSI-Netzwerkschichten und Applikationen zur Verfügung stellen. Die Wiederherstellung des Systems ist dabei ähnlich einfach wie die der Virtuellen Honeypots. Durch die eingeschränkten Möglichkeiten der Emulation von verschiedenen Diensten oder Systemen werden emulierte Honeypots als Low-Ineractive eingestuft. Emulated Honeypots eigenen sich besonders für Honeypot Einsteiger, da sie meist einfach zu Konfigurieren sind und so schnell erste, zumeist automatisierter, Kompromitierungsversuche entdecken.
 
Viele Tools wie z.B. honeyd arbeiten mit diesem Prinzip.
\newpage

 
