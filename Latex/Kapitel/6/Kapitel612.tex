\subsection{Intrusion Detection System}
In Netzwerken werden Intrusion Detection Systeme zur Überwachung und Erkennung von Unregelmäßigkeiten eingesetzt. Im Gegensatz zu einer Firewall, welche erkennbar vor einer DMZ eingesetzt wird, arbeiten IDS im Verborgenen ohne aktiv in das Geschehen einzuschreiten. 
Um den vielfältigen Aufgaben gerecht zu werden, besteht ein IDS aus folgenden vier Modulen:
\begin{itemize}							% Beginn der Aufzählungsumgebung
\item{\textbf{Collection (Sammeln)}} 
\\Für das Sammeln aller relevanten Daten ist das Collection Modul zuständig. Hierbei können zwei verschiedene Vorgehensweisen unterschieden werden. Zum einen gibt es die Host-basierten IDS und zum anderen Netzwerk-basierte Systeme. Soll ein ganzes Netzwerk überwacht werden, eignet sich ein Host basiertes IDS nicht, da hierfür auf jedem Host einzeln ein IDS installiert werden muss. Sinnvoller wäre hier eine Überwachung des Netzwerk-Traffics. Umgesetzt wird das durch sogenannte Sensoren die im Netz eingebunden werden und dort die Daten abgreifen. Möglich ist das an einem Switchport der als Monitoring-Port konfiguriert wird. Ein solches Vorgehen wird Netzwerk basiertes Intrusion Detection System genannt. 
\item{\textbf{Detection (Erkennen)}} 
\\Um einen Angriff erkennen zu können gibt es diverse Möglichkeiten. Ein Pattern basiertes System vergleicht die gesammelten Daten mit gespeicherten Pattern, welche aus bekannten Gefahren bestehen. Sobald es Übereinstimmungen gibt handelt es sich meist um einen Angriff. Nachteil hierbei ist, dass regelmäßig Updates eingespielt werden müssen, ähnlich wie es bei einem Anti-Viren Programm ist.

Eine Alternative ist ein Anomalie-Detection-System. Bei diesem Vorgehen wird in einer ersten Phase der Netzwerk-Verkehr mitgeschnitten und gespeichert. Aus den gesammelten Daten leitet das System einen \glqq normalen\grqq Zustand ab. In der zweiten Phase vergleicht das System den aktuellen Traffic mit dem von ihm als \glqq zulässigen\grqq gespeicherten Traffic und sucht nach Abweichungen. Vorteil hierbei ist, dass auch nicht bekannte Angriffe erkannt werden können.
\item{\textbf{Analyse (Analysieren)}} 
\\Das Analyse Modul bereitet die Ergebnisse des vorherigen Moduls grafisch auf und in gibt Statistiken über die ausgewerteten Daten aus. 
\item{\textbf{Response (Reaktion)}} 
\\Häufig greift ein IDS nicht aktiv in den Netzwerk-Traffic ein um weiterhin unerkannt zu bleiben. Dennoch sollte ein Intrusion Detection System handeln wenn ein verdächtiges Verhalten entdeckt wird. Um das Vorgehen des möglichen Angriffs besser rekonstruieren zu können sollten die verdächtigen Aktivitäten mitgeloggt werden. Bei möglichen kritischen Problemen können unter Umständen auch Alarme, in Form von SMS oder E-Mail an den Administrator, erzeugt werden.
\end{itemize}
Bei genauer Betrachtung der Arbeitsweise zeigt sich, warum ein Honeypot oft auch als Teil eines IDS angesehen wird. Sowohl ein Host basiertes, als auch ein Netzwerk basiertes IDS bietet die Möglichkeit einen Honeypot zu überwachen und das Verhalten eines Angreifers zu dokumentieren. Um Handlungen auf dem Honeypot nachvollziehen zu können eignet sich jedoch ein Host basiertes System deutlich besser, da hier auch die Aktionen auf dem Host geloggt werden können. Die Überwachung des Honeypots gestaltet sich deutlich einfacher als die eines normalen Systems. Denn auch hier gilt jeder Zugriff auf den Honeypot als potentieller Angriff. Somit muss das IDS nicht entscheiden, nach welchen Mustern Angriffe gedeutet werden. 

Abschließend kann festgehalten werden: Für diese Arbeit dient ein IDS aus Sicht eines Honeypots zur Aufzeichnung und Auswertung der Angriffe, sowie zur Alarmierung. Dies schließt ausdrücklich nicht die Ansicht aus, dass ein Honeypot Teil eines IDS sein kann. In diesem Fall kommt es nur auf den Blickwinkel an, aus dem das Thema betrachtet wird.